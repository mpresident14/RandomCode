\documentclass[12pt]{article}

\usepackage{geometry}
\usepackage{amsmath,amsthm,amssymb}
\usepackage{graphicx}
\usepackage{multicol}

\newcommand{\lemma}{\noindent \textbf{Lemma: }}
\newcommand{\thm}{\noindent \textbf{Theorem: }}
\newcommand{\lskip}{\vspace{\baselineskip}}

\begin{document}

\title{Randomized Trees and Treaps}
\author{}
\maketitle


\section*{Description}
We prove below that given a set of $n$ random integers, we can insert them in order into a binary search tree and the expected depth of a node will be $O(\log n)$. This is known as a \emph{random tree}. However, in practice, we don't have all the values that we want to put into the tree up front. For both randomized trees and treaps, the goal is to simulate a random tree by inserting the value at a random index in the existing set of values while using rotations to maintain the search constraints.

Treaps randomly generate a priority for each node and maintain the heap property among the priorities in addition to the search constraints. An equivalent way of describing the treap is that it could be formed by inserting the nodes highest-priority-first into a binary search tree without doing any rebalancing. Since, we generate each priority randomly when adding a node, treap insertion simulates putting a value at a random index in the set.

For a randomized tree, we can think of each node being indexed from 1 to $n$. Given a tree with $n$ nodes, the $n+1^{th}$ node would have had a $\frac{1}{n+1}$ chance of being at index 1 (the root).


\section*{Time Complexity}
Suppose we have a set $S$ of $n$ random numbers that we insert in order into a binary tree. Let $X$ be a random variable representing the number of ancestors of node $x$, i.e., the depth of $x$. Let $X_i$ be a Bernoulli random variable equal to
\[
  \begin{cases}
      1 & \text{ if node $x_i$ is an ancestor of $x$} \\
      0 & \text{otherwise}
   \end{cases}
\]
Then, $X = \sum_{i=1}^nX_i$, so $E[X] = \sum_{i=1}^nE[X_i] = \sum_{i=1}^nP(X_i=1)$. For any $i$, node $x_i$ is an ancestor of $x$ only when $x_i$ is the first element in the range $[x_i, x]$ (or $[x, x_i]$ if $x < x_i$) to be inserted into the tree; otherwise, if some other node $y$ were inserted before $x_i$, $y$ would be at the root of some subtree with $x$ and $x_i$ on opposite branches. If $x_i$'s index in the set is 1 position away from $x$, then there are 2 nodes in the range $[x_i, x]$, so the probability of picking $x_i$ first is $\frac{1}{2}$. Similarly, if $x_i$'s index is 2 positions away, there are 3 nodes in the range, so the probability of picking $x_i$ first is $\frac{1}{3}$. Generalizing, the probability of picking $x_i$ first when $x_i$ is $k$ steps away from $x$ is $\frac{1}{1+k}$. Suppose $x$ is at position $j$ in the set. Then,
\begin{align*}
  E[X] &= \sum_{i=1}^nP(X_i=1) \\
  & = \sum_{i=1}^{j-1} \frac{1}{i+1} + \sum_{i=j+1}^n \frac{1}{(i-j)+1} \\
  &= \sum_{i=1}^{j-1} \frac{1}{i+1} + \sum_{i=1}^{n-j} \frac{1}{i+1} \\
  &\leq 2\sum_{i=1}^{n} \frac{1}{i+1} \\
  &\leq 2\sum_{i=1}^{n} \frac{1}{i} \\
  &\leq 2H_n \\
  & \in O(\log n)
\end{align*}




\end{document}
